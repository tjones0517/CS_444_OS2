\documentclass{article}
\usepackage[utf8]{inputenc}

\usepackage{fullpage}
\usepackage{hyperref}

\title{Assignment 3 - Write-Up}
\author{Tyler Jones, Claude Maimon}
\date{November 15th 2017}

\begin{document}

\maketitle

\newpage
\section{The design you plan to use to implement the LOOK algorithms}
In order to implement an encrypted block device, we will use an existing implementation for the block device and manipulate it to match our requirements. Our block device will encrypt and decrypt the data before writing it to the disk and  after reading it from the disk. We will use the simple block cipher API for the encryption. 
 
\section{Version control log}
 \begin{tabular}{|c |c | c | p{5cm}|} 
 \hline
 Date & Name & Commit & Message \\ [0.5ex] 
 \hline\hline
 11/07 & Tyler & 9a83b6abaebab4d6441c392784ef42c47b524a43 & First attempt at block device driver\\
 \hline
 11/11 & Tyler & a9f5919195a0899f9f5d59e9a061db0361d97e2e &adding makefile and Kconfig, and working version of RAM disk device driver\\
 \hline
  11/11 & Tyler & 21e558fbb963e2bd6e75c413b2cc87b44a5a7a90 & adding mounting instructions \\
 \hline  
  11/11 & Tyler & 4a5b251f8fe67e480ec315cafb0984fc2b4685ca & Update readme.txt\\
 \hline
  11/12 & Claude & b9c18115b5cc3f7631057dbe8b02099ac9e95d2f & Added encryption to block. Made a coppy of the block file\\
 \hline
  11/12 & Claude & 9cb00df11d0c235120e05b41454e43ad2dd3e699 & fixed qemu command.\\
 \hline
  10/15 & Claude & 760ed0d9feedd589f8cd5ab96677962e5aeffe73 &Fixed bugs in tjmc\_a3\_crypto.c\\
 \hline
  11/15 & Claude & 37d95e5af6a8580bd5e6479192d5732871ac9185 &final changes to tjmc\_a3\_crypto.c\\
 \hline 
 
\end{tabular}


\section{A work log}

This assignment was mostly searching for information about the crypto library and the disk driver. The version log shows the actual code changes made but we spent much more time searching for information. We started by reading the old implementation of the kernel given to us in the assignment instructions. After that, we moved on to searching for a good implementation of a RAM disk device driver that we could base our implementation on. Once we implemented the disk driver, we searched the crypto API for a very long time.Since it is not well documented, it took us a few day to understand how to use it properly.

\section{Questions}
\subsection{What do you think the main point of this assignment is?}

We think that the main point of this assignment was to teach us how to use a badly documented API. The crypto API documentation is very unhelpful and examples for using it are hard to understand. There is not man page and you have to read through a lot of information to understand how to use it. 

\subsection{How did you personally approach the problem? Design decisions, algorithm, etc.}

We started by going over the old kernel implementation given to us in the assignment. After that, when we realized we don't need to write our own device driver from scratch we looked for a good implementation of one. We found a simple block driver (http://blog.superpat.com/2010/05/04/a-simple-block-driver-for-linux-kernel-2-6-31/) and we based our disk drive using it. For the encrypting part, we researched the crypto API for a very long time looking at the examples and descriptions. We chose to use the Single Block Cipher API with the AES option. For the implemetation of the encryption, we ise the example code in the Linux Kernel Crypto API (https://kernel.readthedocs.io/en/sphinx-samples/crypto-API.html\#single-block-cipher-api)

\subsection{How did you ensure your solution was correct? Testing details, for instance.}

\subsection{What did you learn?}
We learned how to use a library when it's documentation is very bad. We learned to look at kernel examples and figure out how to use those examples in our implementation. We also learned how to use other peopel's implementation (the disk drive) and use it for out implementation. 

\subsection{How should the TA evaluate your work? Provide detailed steps to prove correctness}
In order to evalute our work, our TA should do the following:

1) First and foremost, create a file called sstf-iosched.c by typing "cp noop-iosched.c sstf-iosched.c" within the block directory. This will create a copy of noop-iosched.c with the proper name that our patch file will now operate on. Do not change the contents within this file, the patch file will do that. Next, run our provided patch file on a clean version of the linux kernel. This patch file should be run within the block directory, and can be run as "patch < sstf.patch". There will be changes to Kconfig.iosched sstf-iosched.c and Makefile, all within the block directory.

2) Next, our TA should compile a new image of the kernel with "make -j4 all" or an equivalent in order to compile the patched changes.

3) Next, our TA should run the VM with the following command:

qemu-system-i386 -gdb tcp::5531 -S -nographic -kernel linux-yocto-3.19/arch/x86/boot/bzImage -drive file=core-image-lsb-sdk-qemux86.ext4 -enable-kvm -net none -usb -localtime --no-reboot --append "root=/dev/hda rw console=ttyS0 debug"

Note that "if=virtio" and "vda" was deleted and changed to "hda" respectively from the command used in assignment 1.

4) Once in the kernel, the TA should do the following:

echo sstf > /sys/block/hda/queue/scheduler

This will change the i/o scheduler from the CFQ default, to sstf, which is our CLOOK implementation.

5) From here, the TA should generate i/o in order to test that our solution is correct. Our code that we changed contains printk statements which will verify visually that requests are being processeed in a CLOOK fashion. They will be sorted, and various print statements will show for various special cases.

A very rudimentary way that we did this was to simply "echo test\_text > test\_io\_file", and to examine the output.

This output is also stored in /var/log/messages.

Essentially, verifying that the request queue is in order from least to greatest was sufficient evidence for proof of our correct implementation.





\end{document}
